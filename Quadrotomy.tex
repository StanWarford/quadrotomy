% Alex Prieger and Stan Warford
% Pepperdine University
% File: Case-Analysis
% !TEX TS-program = xelatex

\documentclass[12pt, fleqn, leqno]{article}

\usepackage{fontspec}
\usepackage[slantedGreek]{mathptmx}
\usepackage{amsmath, amsthm, amssymb,latexsym}
\usepackage{wasysym}                                % For temporal operators, Diamond and Box
\usepackage{eucal}                                  % For temporal operators, Until and Wait
\usepackage{booktabs}                               % For table rules
\usepackage{boxedminipage}                          % For boxing text in Hasse diagram
\usepackage{paralist}
\usepackage{ellipsis}
\usepackage{array}
\usepackage{verbatim}

% For Dr. Warford's computer
%\begin{comment}
\setmainfont{Times}[ItalicFont    ={* Italic},
                    BoldFont      ={* Bold},
                    BoldItalicFont={* Bold Italic}]
\setsansfont{Helvetica}
%\end{comment}

% For Alex's computer
\begin{comment}
\setmainfont{Times New Roman}[ItalicFont    ={* Italic},
                    BoldFont      ={* Bold},
                    BoldItalicFont={* Bold Italic}]
\setsansfont{Arial}
\end{comment}

\newcommand{\lgap}{2pt}                             % Line gap
\newcommand{\llgap}{6pt}                            % Larger line gap
\newcommand{\lllgap}{12pt}                          % Gap between tables
\newcommand{\mymathindent}{24pt}                    % Indentation for math tabbing
\newcommand{\equivs}{\ensuremath{\;\equiv\;}}       % Equivales with space
\newcommand{\equivss}{\ensuremath{\;\;\equiv\;\;}}  % Equivales with double space
\newcommand{\lors}{\ensuremath{\;\lor\;}}           % Disjunction with space
\newcommand{\lorss}{\ensuremath{\;\;\lor\;\;}}      % Disjunction with double space
\newcommand{\lands}{\ensuremath{\;\land\;}}         % Conjunction with space
\newcommand{\landss}{\ensuremath{\;\;\land\;\;}}    % Conjunction with double space
\newcommand{\nequiv}{\ensuremath{\not\equiv}}       % Inequivalent
\newcommand{\impl}{\ensuremath{\Rightarrow}}        % Implies
\newcommand{\nimpl}{\ensuremath{\not\Rightarrow}}   % Does not imply
\newcommand{\foll}{\ensuremath{\Leftarrow}}         % Follows from
\newcommand{\nfoll}{\ensuremath{\not\Leftarrow}}    % Does not follow from

% Macros for Temporal Operators
\newcommand{\Until}{\;\mathcal{U}\;}
\newcommand{\Wait}{\;\mathcal{W}\;}
\newcommand{\Next}{\;\,\text{\raisebox{3.5pt}{\circle{6}}}}
\newcommand{\Event}{\Diamond\,}
\newcommand{\Always}{\Box\,}

\DeclareMathOperator{\divides}{divides}

\newcommand{\myqed}{\rule[-.23ex]{1.2ex}{2.0ex}}
\newcommand{\myqedtab}{\hspace{384pt}}              % For flush right qed symbol in tabbing environment. No longer used.
\newcommand{\spacer}{\vspace{-30pt}}
\newcommand{\firstspacer}{\vspace{-26pt}}

% Thanks to David Gries for sharing the following macros
% Macros for quantifications.
\newcommand{\thedr}{\rule[-.25ex]{.32mm}{1.75ex}}   % Symbol that separates dummy from range in quantification
\newcommand{\dr}{\;\,\thedr\,\;}                    % Symbol that separates dummy from range, with spacing
\newcommand{\rb}{:}                                 % Symbol that separates range from body in quantification
\newcommand{\drrb}{\;\thedr\,{:}\;}                 % Symbol that separates dummy from body when range is missing
\newcommand{\all}{\forall}                          % Universal quantification
\newcommand{\ext}{\exists}                          % Existential quantification

% Macros for proof hints
\newcommand{\Gll} {\langle}                         % Open hint
\newcommand{\Ggg} {\rangle}                         % Close hint
\newlength{\Glllength}                              % Length of open hint symbol
\settowidth{\Glllength}{$.\Gll$}
\newcommand{\Hint}[1]     {\ \ \ $\Gll              \mbox{#1} \Ggg$ }   % Single line hint
\newcommand{\Hintfirst}[1]{\ \ \ $\Gll              \mbox{#1}$ }        % First line of multiline hint
\newcommand{\Hintmid}[1]  {\ \ $\hspace{\Glllength} \mbox{#1}$ }        % Middle line of multiline hint
\newcommand{\Hintlast}[1] {\ \ $\hspace{\Glllength} \mbox{#1} \Ggg$ }   % Last line of multiline hint

% Single and double quotes
\newcommand{\Lq}{\mbox{`}}
\newcommand{\Rq}{\mbox{'}}
\newcommand{\Lqq}{\mbox{``}}
\newcommand{\Rqq}{\mbox{''}}

\DeclareMathOperator{\incomp}{incomp}

\oddsidemargin  0.0in
\evensidemargin 0.0in
\textwidth      6.0in
\headheight     0.0in
\topmargin      0.0in
\textheight=8.8in
%\parindent=0in
%\pagestyle{plain}

\pagestyle{myheadings} 
\markboth{\textbf{Draft} (\today)} {\textbf{Draft} (\today)}

\title{A Quadrotomy for Partial Orders}

\author{
   Alex Prieger\\
   J. Stanley Warford\\
   Pepperdine University\\
   Malibu, CA 90263}
\date{} % Required for no date to appear in heading

\begin{document}
\maketitle
\begin{abstract}

This paper uses a calculational system to formally prove a quadrotomy for partial orders $\preceq$, which states that any element $x$ of the partial order must be related to any other element $y$ by exactly one of (a) $x \prec y$, (b) $x=y$, (c) $y\prec x$, or (d) $\incomp(x,y)$, where $\prec$ is the reflexive reduction of $\preceq$ and $\incomp(x,y)$ is defined as $\incomp(x,y) \equivs \neg (x\preceq y) \lands \neg (y\preceq x)$.
The calculational system, developed by Dijkstra and Scholten and extended by Gries and Schneider in their text \textit{A Logical Approach to Discrete Math}, is based on only four inference rules -- Substitution, Leibniz, Equanimity, and Transitivity. 
Inference rules in the older Hilbert-style systems, notably modus ponens, appear as theorems in this calculational deductive system, which is used to prove algorithm correctness in computer science.
The theorem presented in this paper is a generalization of the trichotomy theorem for integers proved by Gries and Schneider, which states that any integer $n$ must be related to any other integer $m$ by exactly one of (a) $n<m$, (b) $n=m$, or (c) $m<n$.

\end{abstract}

\thispagestyle{plain}

\section{Introduction}

\subsection{Background}

Propositional calculus is a formal system of logic based on the unary operator negation $\neg$,
the binary operators conjunction $\land$, disjunction $\lor$, implies $\impl$ (also written $\rightarrow$),
and equivalence $\equiv$ (also written $\leftrightarrow$),
variables (lowercase letters $p$, $q$, \dots), and the constants \textit{true} and \textit{false}.
Hilbert-style logic systems, $\mathcal{H}$, are the deductive logic systems traditionally used in mathematics to describe the propositional calculus.
A key feature of such systems is their multiplicity of inference rules and the importance of modus ponens as one of them.

In the late 1980's, Dijkstra and Scholten \cite{DandS}, and Feijen \cite{Feij} developed a method of proving program correctness with a new logic based on an equational style.
In contrast to $\mathcal{H}$ systems, $\mathcal{E}$ has only four inference rules -- Substitution, Leibniz, Equanimity, and Transitivity.
In $\mathcal{E}$, modus ponens plays a secondary role.
It is not an inference rule, nor is it assumed as an axiom, but instead is proved as a theorem from the axioms using the inference rules.

Gries and Schneider \cite{Gries1995, Gries1995145} show that $\mathcal{E}$, also known as a \textit{calculational} system, has several advantages over traditional logic systems.
The primary advantage of $\mathcal{E}$ over $\mathcal{H}$ systems is that the calculational system has only four proof rules, with inference rule Leibniz as the primary one.
Roughly speaking, Leibniz is ``substituting equals for equals,'' hence the moniker \textit{equational} deductive system.
In contrast, $\mathcal{H}$ systems rely on a more extensive set of inference rules.

In 1994, Gries and Schneider published \textit{A Logical Approach to Discrete Math} (LADM) \cite{LADM}, in which they first develop $\mathcal{E}$ for propositional and predicate calculus, and then extend it to a theory of sets, a theory of sequences, relations and functions, a theory of integers, recurrence relations, modern algebra, and a theory of graphs.
Using calculational logic as a tool, LADM brings all the advantages of $\mathcal{E}$ to these additional knowledge domains.
The treatment is in marked contrast to the traditional one exemplified by the classic undergraduate text by Rosen \cite{Rosen}.

\subsection{Partial Orders}

Paragraph or section on this application of $\mathcal{E}$ to the problem.
Reference to LADM trichotomy.

\section{Results}

\subsection{Quadrotomy}

Overall plan.

Lemma 1: $b \prec c \lors b = c \equivss b \preceq c$, where $\prec$ is the reflexive reduction of $\preceq$.

\textit{Proof}:
\begin{tabbing}
\hspace{\mymathindent} \= $= \;$ \= \myqedtab \= \kill
	\> \>  $b \prec c \lors b = c$\\
	\> $=$  \>  \Hint{(14.15.4) Notation, $ \langle b, c \rangle \in \rho$ and $b \; \rho \; c$ are interchangeable notations.}\\[\lgap]
	\> \>   $ \langle b, c \rangle \in\; \prec \lors b = c$\\
	\> $=$  \>  \Hint{(14.15.3) Identity lemma, $ \langle x, y \rangle \in\; i_B \equivs x = y$}\\[\lgap]
	\> \>   $ \langle b, c \rangle \in\; \prec \lors \langle b, c \rangle \in i_B$\\
	\> $=$  \>  \Hint{(11.20) Axiom, Union, $v \in\; S \cup T \equivss v \in S \lors v \in T$}\\[\lgap]
	\> \>   $ \langle b, c \rangle \in\; \prec \cup\; i_B$\\
	\> $=$  \>  \Hint{(14.49b) If $\rho$ is a quasi order over a set B, then $\rho \cup i_B$ is a partial order}\\[\lgap]
	\> \>   $ \langle b, c \rangle \in\; \preceq$\\
	\> $=$  \>  \Hint{(14.15.4) Notation}\\[\lgap]
	\> \>   $b \preceq c$\quad \myqed\\
\end{tabbing}

Lemma 2: $(p \lors q) \lands \lnot (q \lors r) \equivss p \lands \lnot q \lands \lnot r$

\textit{Proof}:
\begin{tabbing}
\hspace{\mymathindent} \= $= \;$ \= \myqedtab \= \kill
	\> \>  $(p \lors q) \lands \lnot (q \lors r)$\\
	\> $=$  \>  \Hint{(3.47b) De Morgan, $\lnot (p \lors q) \equivss \lnot p \lands \lnot q$}\\[\lgap]
	\> \>   $(p \lors q) \lands \lnot q \lands \lnot r$\\
	\> $=$  \>  \Hintfirst{(3.44a) Absorption, $p \lands (\lnot p \lors q) \equivss (p \lands q)$, with $p,\;q:= \lnot q,\;p$ and with}\\
	\>			 \>  \Hintlast{(3.12) Double negation}\\[\lgap]
	\> \>   $p \lands \lnot q \lands \lnot r$ \quad \myqed\\
\end{tabbing}

Lemma 3: $(p \impl q) \impl (p \lands q \equivs p)$

\textit{Proof}:
\begin{tabbing}
\hspace{\mymathindent} \= $= \;$ \= \myqedtab \= \kill
	\> \>  $p \lands q \equivs p$\\
	\> $=$  \>  \Hint{(3.60) Implication, $p \impl q \equivss p \lands q \equivs p$}\\[\lgap]
	\> \>   $p \impl q$\\
	\> $\foll$  \>  \Hint{(3.71) Reflexivity of $\impl$, $p \impl p$}\\[\lgap]
	\> \>   $p \impl q$ \quad \myqed\\
\end{tabbing}

Lemma 4: $\rho$ is irreflexive $\impl (b \rho c \impl \lnot (b = c))$

\textit{Proof}: The proof is by (4.4) Deduction.
\begin{tabbing}
\hspace{\mymathindent} \= $= \;$ \= \myqedtab \= \kill
	\> \>  $b \rho c \impl \lnot (b = c)$\\
	\> $=$  \>  \Hint{(3.61) Contrapositive, $p \impl q \equivs \lnot q \impl \lnot p$, with (3.12) Double negation}\\[\lgap]
	\> \>   $b = c \impl \lnot (b \rho c)$\\
	\> $=$  \>  \Hint{(3.84b) Substitution, $(e = f) \impl E^z_e \equivs (e = f) \impl E^z_f$}\\[\lgap]
	\> \>   $b = c \impl \lnot (b \rho b)$\\
	\> $=$  \>  \Hint{Assume the antecedent, $\rho$ is irreflexive, or $(\all b \dr \rb \lnot (b \rho b))$}\\[\lgap]
	\> \>   $b = c \impl true$\\
	\> $=$  \>  \Hint{(3.72) Right Zero of $\impl$, $p \impl true \equivs true$}\\[\lgap]
	\> \>   $true$ \quad \myqed\\
\end{tabbing}

Lemma 5: $\rho$ is antisymmetric $\lands$ $\rho$ is reflexive $\impl (b \rho c \lands c \rho b \equivs b = c)$

\textit{Proof}: The proof is by (4.4) Deduction, \textit{i.e.}, prove the consequent, $b \rho c \lands c \rho b \equivs b = c$, assuming the conjuncts of the antecedent.

Using (4.7) Mutual implication, the proof of $b \rho c \lands c \rho b \impl b = c$ follows.

\begin{tabbing}
\hspace{\mymathindent} \= $= \;$ \= \myqedtab \= \kill
	\> \>  $b \;\rho\; c \lands c \;\rho\; b$\\
	\> $\impl$  \>  \Hintfirst{Assume the conjunct of the antecedent $\rho$ is antisymmetric, or}\\
	\>			 \>  \Hintlast{$(\all b,c \drrb b \;\rho\; c \lands c \;\rho\; b \impl b = c)$}\\[\lgap]
	\> \>   $b = c$ \quad \myqed\\
\end{tabbing}

The proof of $b = c \impl b \rho c \lands c \rho b$ is by (4.4) Deduction.

\textit{Proof}:
\begin{tabbing}
\hspace{\mymathindent} \= $= \;$ \= \myqedtab \= \kill
	\> \>  $b \;\rho\; c \lands c \;\rho\; b$\\
	\> $=$  \>  \Hint{Assume the antecedent, $b = c$}\\[\lgap]
	\> \>   $b \;\rho\; b \lands b \;\rho\; b$\\
	\> $=$  \>  \Hint{(3.38) Idempotency of $\land$, $p \lands p \equivs p$}\\[\lgap]
	\> \>   $b \;\rho\; b$\\
	\> $=$  \>  \Hint{Assume the conjunct of the antecedent $\rho$ is reflexive, or $(\all b \drrb b \;\rho\; b)$}\\[\lgap]
	\> \>   $true$ \quad \myqed\\
\end{tabbing}

Theorem, Quadrotomy (a): $b \prec c \equivss \lnot(c \prec b \lors b = c \lors \incomp(b, c))$

\textit{Proof}:
\begin{tabbing}
\hspace{\mymathindent} \= $= \;$ \= \myqedtab \= \kill
	\> \>  $\lnot(c \prec b \lors b = c \lors \incomp(b, c))$\\
	\> $=$  \>  \Hint{Lemma 1, $b \prec c \lors b = c \equivss b \preceq c$}\\[\lgap]
	\> \>   $\lnot(c \preceq b \lors \incomp(b, c))$\\
	\> $=$  \>  \Hint{(14.47.1) Definition, Incomparable, $\incomp(b, c) \equivss \lnot (b \preceq c) \lands \lnot (c \preceq b)$}\\[\lgap]
	\> \>   $\lnot (c \preceq b \lors (\lnot (b \preceq c) \lands \lnot (c \preceq b)))$\\
	\> $=$  \>  \Hint{(3.44b) Absorption, $p \lors (\lnot p \lands q) \equivss p \lors q$}\\[\lgap]
	\> \>   $\lnot (c \preceq b \lors \lnot (b \preceq c))$\\
	\> $=$  \>  \Hint{(3.47b) De Morgan, $\lnot (p \lors q) \equivs \lnot p \lands \lnot q$, with (3.12) Double negation}\\[\lgap]
	\> \>   $b \preceq c \lands \lnot(c \preceq b)$\\
	\> $=$  \>  \Hint{Lemma 1, twice}\\[\lgap]
	\> \>   $(b \prec c \lors b = c) \landss \lnot (c \prec b \lors b = c)$\\
	\> $=$  \>  \Hint{Lemma 2, $(p \lors q) \lands \lnot (q \lors r) \equivss p \lands \lnot q \lands \lnot r$}\\[\lgap]
	\> \>   $b \prec c \lands \lnot (b = c) \lands \lnot (c \prec b)$\\
	\> $=$  \>  \Hintfirst{Lemma 3, $(p \impl q) \impl (p \lands q \equivs p)$, with $p,\;q := b \prec c,\;\lnot (c \prec b)$. The antecedent}\\
	\>			 \>  \Hintlast{is true because strict orders are asymmetric.}\\[\lgap]
	\> \>  $b \prec c \lands \lnot (b = c)$\\
	\> $=$  \>  \Hintfirst{Lemma 3, with $p,\;q := b \prec c,\;\lnot (b = c)$. The antecedent, $b \prec c \impl \lnot (b = c)$, is}\\
	\>			 \>  \Hintmid{true by Lemma 4, $\rho$ is irreflexive $\impl (b \;\rho\; c \impl \lnot (b = c))$, and the fact that strict}\\
	\>			 \>  \Hintlast{orders are irreflexive.}\\[\lgap]
	\> \>   $b \prec c$ \quad \myqed\\
\end{tabbing}

Theorem, Quadrotomy (b): $c \prec b \equivss \lnot(b \prec c \lors b = c \lors \incomp(b, c))$

Because = and \textit{incomp} are symmetric, Quadrotomy (b) is simply Quadrotomy (a) with $b,\;c := c,\;b$.\\

Theorem, Quadrotomy (c): $b = c \equivss \lnot(b \prec c \lors c \prec b \lors \incomp(b, c))$

\textit{Proof}:
\begin{tabbing}
\hspace{\mymathindent} \= $= \;$ \= \myqedtab \= \kill
	\> \>  $\lnot (b \prec c \lors c \prec b \lors \incomp(b, c))$\\
	\> $=$  \>  \Hint{(3.26) Idempotency of $\lor$, $p \equivs p \lors p$}\\[\lgap]
	\> \>   $\lnot (b \prec c \lors c \prec b \lors \incomp(b, c) \lors \incomp(b, c))$\\
	\> $=$  \>  \Hint{(14.48.2), $\lnot (b \preceq c) \equivss c \prec b\lors \incomp(b, c)$, twice}\\[\lgap]
	\> \>   $\lnot ( \lnot (b \preceq c) \lors \lnot (c \preceq b))$\\
	\> $=$  \>  \Hint{(3.47a) De Morgan, $\lnot (p \lands q) \equivs \lnot p \lors \lnot q$, with (3.12) Double negation}\\[\lgap]
	\> \>   $b \preceq c \lands c \preceq b$\\
	\> $=$  \>  \Hintfirst{Lemma 5, $\rho$ is antisymmetric $\lands$ $\rho$ is reflexive $\impl (b \;\rho\; c \lands c \;\rho\; b \equivss b = c)$,}\\
	\>			 \>  \Hintlast{with the fact that partial orders are antisymmetric and reflexive}\\[\lgap]
	\> \>   $b = c$ \quad \myqed\\
\end{tabbing}

Theorem, Quadrotomy (d): $\incomp(b, c) \equivss \lnot(b \prec c \lors c \prec b \lors b = c)$

\textit{Proof}:
\begin{tabbing}
\hspace{\mymathindent} \= $= \;$ \= \myqedtab \= \kill
	\> \>  $\lnot(b \prec c \lors c \prec b \lors b = c)$\\
	\> $=$  \>  \Hint{(3.26) Idempotency of $\lor$, $p \equivs p \lors p$}\\[\lgap]
	\> \>   $\lnot(b \prec c \lors c \prec b \lors b = c \lors b = c)$\\
	\> $=$  \>  \Hint{Lemma 1, $b \prec c \lors b = c \equivss b \preceq c$ twice}\\[\lgap]
	\> \>   $\lnot (b \preceq c \lors c \preceq b)$\\
	\> $=$  \>  \Hint{(3.47b) De Morgan, $\lnot (p \lors q) \equivs \lnot p \lands \lnot q$}\\[\lgap]
	\> \>   $\lnot (b \preceq c) \lands \lnot (c \preceq b)$\\
	\> $=$  \>  \Hint{(14.47.1) Definition, Incomparable: $\incomp(b, c) \equivss \lnot(b \preceq c) \lands \lnot(c \preceq b)$}\\[\lgap]
	\> \>   $\incomp(b, c)$ \quad \myqed\\
\end{tabbing}

\subsection{Generalized Trichotomy}

Overall plan.

Corollary to (14.50): $\incomp(b,c) \equivs false$, where $\incomp(b,c)$ refers to b and c being incomparable under a total order $\le$.

\textit{Proof}:
\begin{tabbing}
\hspace{\mymathindent} \= $= \;$ \= \myqedtab \= \kill
	\> \>  $\incomp(b,c)$\\
	\> $=$  \>  \Hintfirst{(14.47.1) Definition, Incomparable, $\incomp(b,c)  \equivss  \lnot (b \preceq c) \lands  \lnot (c \preceq b)$,}\\
	\>			\>  \Hintlast{with $\preceq := \le$}\\[\lgap]
	\> \>   $\lnot (b \le c) \lands \lnot (c \le b)$\\
	\> $=$  \>  \Hint{(3.47b) De Morgan, $\lnot (p \lors q) \equivs \lnot p \lands \lnot q$}\\[\lgap]
	\> \>   $\lnot (b \le c \lors c \le b)$\\
	\> $=$  \>  \Hintfirst{(14.50) Definition, Total Order: A partial order $\preceq$ over \textit{B} is called a total or}\\
	\>			\>  \Hintlast{linear order if $(\all b,c \drrb b \preceq c \lors c \preceq b)$}\\[\lgap]
	\> \>   $\lnot true$\\
	\> $=$  \>  \Hint{(3.8) Definition of false}\\[\lgap]
	\> \>   $false$\quad \myqed\\
\end{tabbing}

Theorem, Trichotomy (a): $b < c \equivss \lnot(c < b \lors b = c)$, where < is a strict total order.

\textit{Proof}:
\begin{tabbing}
\hspace{\mymathindent} \= $= \;$ \= \myqedtab \= \kill
	\> \>  $b < c$\\
	\> $=$  \>  \Hintfirst{(Theorem, Quadrotomy (a), $b \prec c \equivss \lnot(c \prec b \lors b = c \lors \incomp(b, c))$, with}\\
	\>			 \>  \Hintlast{$\prec := <$}\\[\lgap]
	\> \>   $\lnot(c < b \lors b = c \lors \incomp(b, c))$\\
	\> $=$  \>  \Hint{Corollary to (14.50), $\incomp(b,c) \equivs false$ for a total order}\\[\lgap]
	\> \>   $\lnot(c < b \lors b = c \lors false)$\\
	\> $=$  \>  \Hint{(3.30) Identity of $\lor$, $p \lors false \equivs false$}\\[\lgap]
	\> \>   $\lnot(c < b \lors b = c)$\quad \myqed\\
\end{tabbing}

Theorem, Trichotomy (b): $c < b \equivss \lnot(b < c \lors b = c)$, where < is a strict total order.

Trichotomy (b) is simply Trichotomy (a) with $b,c:=c,b$.\\[\lgap]

Theorem, Trichotomy (c): $b = c \equivss \lnot(b < c \lors c < b)$, where < is a strict total order.

\textit{Proof}:
\begin{tabbing}
\hspace{\mymathindent} \= $= \;$ \= \myqedtab \= \kill
	\> \>  $b = c$\\
	\> $=$  \>  \Hintfirst{(Theorem, Quadrotomy (c), $b = c \equivss \lnot(b \prec c \lors c \prec b \lors \incomp(b, c))$, with}\\
	\>			 \>  \Hintlast{$\prec := <$}\\[\lgap]
	\> \>   $\lnot(b < c \lors c < b \lors \incomp(b, c))$\\
	\> $=$  \>  \Hint{Corollary to (14.50), $\incomp(b,c) \equivs false$ for a total order}\\[\lgap]
	\> \>   $\lnot(b < c \lors c < b \lors false)$\\
	\> $=$  \>  \Hint{(3.30) Identity of $\lor$, $p \lors false \equivs false$}\\[\lgap]
	\> \>   $\lnot(b < c \lors c < b)$\quad \myqed\\
\end{tabbing}

\subsection{Gries and Schneider's Trichotomy}

It is not easy to concisely describe a stituation in which several propositions are both mutually exclusive (that is, at most one can be true) and collectively exhaustive (that is, at least one must be true) using only binary boolean operators.
There are several different ways to do so, and differences in the ways chosen account for most of the difficulty in showing that Gries and Schneider's Trichotomy is a special case of our Quadrotomy.

Figure \ref{fig:meceFig} below shows three different ways to do so.
The first we name "Double Implication," since it can be intuitively read as a series of two-way implications, "if and only if" statements.
This is the format we use for the Quadrotomy and the Generalized Trichotomy.

The second we name "Equivalence String," since the core element is a string of equivalences, followed by a string of excluded specific cases.
This is the format Gries and Schneider use for their Trichotomy.

The third we name "Truth Table" because of its strong resemblance to a truth table; each disjunct says that one particular row of the truth table is true.
This is used in none of the theorems, but is an important step in the proof that they are equivalent.

\begin{table}
 \begin{tabular}{ c|c|c|c }
  \toprule
  Format & 2 Propositions & 3 Propositions & 4 Propositions\\
  \midrule
  Double Implication & \begin{tabular}{c} $(p \equiv \lnot q) \lands$ \\ $(q \equiv \lnot p)$\end{tabular} & \begin{tabular}{c} $(p \equiv \lnot(q \lor r)) \lands$ \\ $(q \equiv \lnot(p \lor r)) \lands$ \\ $(r \equiv \lnot(p \lor q))$\end{tabular} & \begin{tabular}{c} $(p \equiv \lnot(q \lor r \lor s)) \lands$ \\ $(q \equiv \lnot(p \lor r \lor s)) \lands$ \\ $(r \equiv \lnot(p \lor q \lor s)) \lands$ \\ $s \equiv \lnot(p \lor q \lor r)$\end{tabular} \\
  \midrule
  Equivalence String & $p \nequiv q$ & \begin{tabular}{c} $(p \equiv q \equiv r) \lands$ \\ $\lnot(p \land q \land r)$\end{tabular} & \begin{tabular}{c} $(p \nequiv q \nequiv r \nequiv s) \lands$ \\ $\lnot(\lnot p \land q \land r \land s) \lands$ \\ $\lnot(p \land \lnot q \land r \land s) \lands$ \\ $\lnot(p \land q \land \lnot r \land s) \lands$ \\ $\lnot(p \land q \land r \land \lnot s)$\end{tabular}\\
  \midrule
  Truth table & \begin{tabular}{c}$ (p \land \lnot q) \lors$ \\ $(\lnot p \land q)$\end{tabular} & \begin{tabular}{c} $(p \land \lnot q \land \lnot r) \lors$ \\ $(\lnot p \land q \land \lnot r) \lors$ \\ $(\lnot p \land \lnot q \land r)$\end{tabular} & \begin{tabular}{c} $(p \land \lnot q \land \lnot r \land \lnot s) \lors$ \\ $(\lnot p \land q \land \lnot r \land \lnot s) \lors$ \\ $(\lnot p \land \lnot q \land r \land \lnot s) \lors$ \\ $(\lnot p \land \lnot q \land \lnot r \land  s)$\end{tabular}\\
  \bottomrule
 \end{tabular}
 \caption{Different ways to express mutually exclusive, collectively exhaustive propositions\label{fig:meceFig}}
\end{table}

It is important to recognize that while the expressions as given may look significantly different, all of the rows are equivalent because they describe the same truth table, which we have checked with an automated truth-table generator.
This can also be proven with the calculational system, but the proof is tedious, and will not be presented here.
Lemma 6 follows from this table—in particular, from the equivalence of the second and third elements of the third column.

Lemma 6:\\ $(p \equiv q \equiv r) \lands \lnot(p \land q \land r) \equivss (p \equivs \lnot(q \lor r)) \lands (q \equivs \lnot(p \lor r)) \lands (r \equivs \lnot(p \lor q))$\\
Lemma 6 is stated without proof, since as mentioned the proof is tedious.\\

(15.44) Trichotomy: $(a < b \lors a = b \lors a > b) \lands \lnot(a < b \lors a = b \lors a > b)$

\textit{Proof}
\begin{tabbing}
\hspace{\mymathindent} \= $= \;$ \= \myqedtab \= \kill
	\> \>  $(a < b \lors a = b \lors a > b) \lands \lnot(a < b \lors a = b \lors a > b)$\\
	\> $=$  \>  \Hint{Lemma 6}\\[\lgap]
	\> \>   $(a < b \equivs \lnot(b < a \lors a = b)) \lands (b < a \equivs \lnot(a < b \lors a = b))$\\
	\> \>   $\lands (a = b \equivs \lnot(a < b \lors b < a))$\\
	\> $=$  \>  \Hintfirst{Theorem, Trichotomy (a), $b < c \equivss \lnot(c < b \lors b = c)$, with $<$ (generic strict}\\
	\>			 \>  \Hintlast{total order) replaced by $<$ (integers)}\\[\lgap]
	\> \>   $true \lands (b < a \equivs \lnot(a < b \lors a = b)) \lands (a = b \equivs \lnot(a < b \lors b < a))$\\
	\> $=$  \>  \Hint{Theorem, Trichotomy (b): $c < b \equivss \lnot(b < c \lors b = c)$, with $<:=<$}\\[\lgap]
	\> \>   $true \lands true \lands (a = b \equivs \lnot(a < b \lors b < a))$\\
	\> $=$  \>  \Hint{Theorem, Trichotomy (c): $b = c \equivss \lnot(b < c \lors c < b)$, with $<:=<$}\\[\lgap]
	\> \>   $true \lands true \lands true$\\
	\> $=$  \>  \Hint{(3.39) Identity of $\land$, $p \land true \equivs p$, twice}\\[\lgap]
	\> \>   $true$\quad \myqed\\
\end{tabbing}

\section{Conclusion}

Something

\bibliographystyle{plain}
\bibliography{myBiblio.bib}

\end{document}


